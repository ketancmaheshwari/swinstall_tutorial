\documentclass{report}

%\usepackage{palatino}
\pagenumbering{gobble}
\usepackage[english]{babel}
\usepackage[pdftex]{color,graphicx}
\usepackage{enumitem}
\setitemize{noitemsep,topsep=0pt,parsep=0pt,partopsep=0pt}
\setenumerate{noitemsep,topsep=0pt,parsep=0pt,partopsep=0pt}
\setlist[itemize]{leftmargin=*,noitemsep}
\usepackage[T1]{fontenc}

\title{ Title\\
       \large A tutorial proposal for PEARC'21
			 }

\author{Ketan Maheshwari}

\begin{document}
\maketitle
\section*{Abstract}

\section*{Detailed description}
\subsection*{Tutorial goals}
\begin{enumerate}
\item Share knowledge of 
\end{enumerate}

\subsection*{Relevance of topic to conference attendees}
I strongly believe the topic is relevant and useful to conference attendees. Consider the following user scenarios:
\begin{itemize}
\item A researcher 
\end{itemize}

\subsection*{Targeted audience}
\subsection*{Content level (\% beginner, \% intermediate, \% advanced)}
80 \% Beginner, 10 \% intermediate, 10 \% advanced
\subsection*{Audience prerequisites}

\subsection*{General description of tutorial content}
Through this tutorial, we  intend to demonstrate and impart the knowledge of

\section*{Detailed outline of the tutorial}
\begin{enumerate}
\item Overview and Logistics
  \begin{enumerate}
    \item Slides and Practice data Download
  \end{enumerate}
\item Linux Basics
  \begin{enumerate}
    \item Anatomy of a Linux command
  \end{enumerate}
\item Streams, Pipes and Redirection
  \begin{enumerate}
    \item The pipe: run second command using the output of first
  \end{enumerate}
\item Classic Productivity Tools
  \begin{enumerate}
    \item The versatile find
  \end{enumerate}
\item Safe and Secure Use of Facilities
  \begin{enumerate}
    \item ssh: commands and config
  \end{enumerate}
\item Session Management with Screen and Tmux
  \begin{enumerate}
    \item About screen and tmux
  \end{enumerate}
\item Scripting for HPC
  \begin{enumerate}
    \item Basics
%   \item Should I write a script or a program
  \end{enumerate}
\item Miscellaneous Productivity Tools
  \begin{enumerate}
    \item bc : a versatile math library and language
  \end{enumerate}
\item Summary and Conclusions
  \begin{enumerate}
    \item A cheatsheet to take home
  \end{enumerate}
\end{enumerate}

\subsection*{A statement about ``hands-on'' exercises}
Each section in the tutorial will be followed by hands-on exercises. The hands on exercises are based on:
\begin{enumerate}
\item The data already available on a
typical Linux system such as files and processes
\item Based on the two data
files that are supplied with the tutorial.
\end{enumerate}
We will provide the solutions to hands-on exercises to students at the end of the tutorial.

\subsection*{Resume or CV for each presenter (Make sure this includes a list of short courses each presenter has taught)}
Attached.
\subsection*{A statement agreeing to release the notes for the SC18 tutorial digital copy}
We agree to release the notes for the SC18 tutorial digital copy.

\end{document}

Abstract

Introduction

Plan

Audience and Benefits


user space scientific software installation

yum, apt, dnf etc. are ok on VMs and Desktops but not on clusters. Why?

Understanding a Makefile and make commands
VERBOSE and parallel with -j
troubleshoot

autotools autoconf
    troubleshoot

configure; make; make install
    troubleshoot

ccmake, cmake, make && make install
    troubleshoot

Environment variables

Compilers times Libraries times versions

update / upgrade software

The python zoo
    2 vs 3
    pip / pip3
    conda
    setup.py
    requirements
    wheel
    environments
    virtual env
    update / upgrade

userspace rpms
    rare but possible

dangers and pitfalls
system-level libraries
best practices

spack and easyinstall

java ant apache build

binary installations

System architecture considerations

Other systems:
npm
rubygems
luarocks

