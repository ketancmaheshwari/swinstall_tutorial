\documentclass{report}

\pagenumbering{gobble}
\usepackage[english]{babel}
\usepackage[pdftex]{color,graphicx}
\usepackage{enumitem}
\setitemize{noitemsep,topsep=0pt,parsep=0pt,partopsep=0pt}
\setenumerate{noitemsep,topsep=0pt,parsep=0pt,partopsep=0pt}
\setlist[itemize]{leftmargin=*,noitemsep}
\usepackage[T1]{fontenc}

\title{ Title\\
       \large A tutorial proposal for PEARC'21
			 }

\author{Ketan Maheshwari}

\begin{document}
\maketitle
\section*{Abstract}
In this tutorial, I propose to offer a comprehensive, definitive, clarifying
tutorial on the practice and process of scientific software installation in the
user-space on large-scale computational infrastructures.

\section*{Detailed description}
Scientific software installation process is often taken for granted. Due to
this, the installations often are suboptimized and results in an
underperforming software and often vulnerable to complex configurations, patchy
or outright erroneous behavior.  In particular with the advent of python
ecosystem, and a number of supported install methods the waters have muddied
quite a bit that often makes users resisting installing software or face
difficulties after a faulty installation.

\subsection*{Tutorial goals}
\begin{enumerate}
\item Share knowledge of scientific software installation in user space over compute clusters and supercomputers
\item Demonstrate best practices for software installation 
\item Clarify the concepts, principles and patterns found in the software installation process
\end{enumerate}

\subsection*{Relevance of topic to conference attendees}
I believe the topic is relevant and useful to conference attendees. The following scenarios support my belief:
\begin{itemize}
\item A researcher experimenting with several versions of a software library would like to efficiently install them in their home area.
\item A scientist who extends a popular scientific software and would like to link it to a cluster installed library.
\item Users who would like to use specialized compilers such as NAG or PGI that are not always available systemwide.
\item Users who would like to supplement the system installed libraries for GPUs with their own Kokkos driven codes.
\end{itemize}

\subsection*{Targeted audience}
\begin{itemize}
\item System administrators and managers who might want to re-use the tutorial for the users of their home cluster environment.
\item Science users will benefit with the clear understanding of the process of software installation crystalized and clarified.
\end{itemize}
\subsection*{Content level (\% beginner, \% intermediate, \% advanced)}
80 \% Beginner, 10 \% intermediate, 10 \% advanced
\subsection*{Audience prerequisites}
Basic knowledge and hands-on experience on Linux command-line. Understanding of
basic Linux concepts such as paths, files and directories, permissions etc.

\section*{Detailed outline of the tutorial}
\begin{enumerate}
\item Overview and Logistics
  \begin{enumerate}
    \item Motivation: why yum and apt does not work on my HPC Environment
    \item Slides and Practice Software Packages Download
  \end{enumerate}
\item The Make Basics
  \begin{enumerate}
    \item Anatomy
  \end{enumerate}
\item The configure, make, make test, make install workflow
  \begin{enumerate}
    \item Configure Concept
    \item Configure Script
    \item Configure Examples
    \item Make and Make Install nitty-gritty
    \item Post-install setup
  \end{enumerate}
\item The ccmake, cmake, make workflow
    \begin{enumerate}
      \item cmake Fundamentals
      \item cmake Troubleshooting
      \item cmake examples
    \end{enumerate}
\item Compilers times Libraries times Versions Combinatorics
  \begin{enumerate}
    \item How to manage efficiently
    \item Updates and upgrades
  \end{enumerate}
\item The Python Zoo
  \begin{enumerate}
    \item 2 vs 3 implications
    \item pip / pip3 and conda
    \item setup.py and wheel
    \item requirements.txt
    \item environment and virtual env
    \item update / upgrade
    \item Tips and Tricks
  \end{enumerate}
\item Other Software Installation Methods
    \begin{enumerate}
        \item spack and easy\_install
        \item rpm in userspace
        \item rubygems, npm, luarocks, cargo
        \item java and apache ant 
    \end{enumerate}
\item Miscellaneous Tips
  \begin{enumerate}
    \item Use session managers for long builds
    \item Save logs and successful commands
  \end{enumerate}
\item Summary and Conclusions
  \begin{enumerate}
    \item A handy cheatsheet to take home
  \end{enumerate}
\end{enumerate}

\subsection*{A statement about ``hands-on'' exercises}
Each section in the tutorial will be followed by hands-on exercises. The hands on exercises are based on:
\begin{enumerate}
\item The data already available on a
typical Linux system such as files and processes
\item Based on the two data
files that are supplied with the tutorial.
\end{enumerate}
We will provide the solutions to hands-on exercises to students at the end of the tutorial.

\subsection*{Resume or CV for each presenter (Make sure this includes a list of short courses each presenter has taught)}
Attached.
\subsection*{A statement agreeing to release the notes for the SC18 tutorial digital copy}
We agree to release the notes for the SC18 tutorial digital copy.

\end{document}

Introduction

Plan

Audience and Benefits

user space scientific software installation

yum, apt, dnf etc. are ok on VMs and Desktops but not on clusters. Why?

Understanding a Makefile and make commands
VERBOSE and parallel with -j
troubleshoot

autotools autoconf
    troubleshoot

configure; make; make install
    troubleshoot

ccmake, cmake, make && make install
    troubleshoot

Environment variables

Compilers times Libraries times versions

update / upgrade software

The python zoo

userspace rpms
    rare but possible

dangers and pitfalls
system-level libraries
best practices

java ant apache build

binary installations

System architecture considerations

Other systems:
npm
rubygems
luarocks

